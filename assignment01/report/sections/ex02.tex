\section*{Exercise 2}
\subsection*{Wave Speed}
We assume that the solution to the finite difference scheme is rightward traveling wave
\begin{equation}
	v(x,t) = e^{ik(x-c_6t)}
	\label{eq:wave_sol}
\end{equation}
We plug \eqref{eq:wave_sol} into our finite difference scheme \eqref{eq:final}
\begin{equation}
	\begin{aligned}
		-ikc_6 e^{ik(x_j-c_6t)} & = -\frac{c}{60 \Delta x} [-e^{ik(x_{j-3}-c_6t)} + 9e^{ik(x_{j-2}-c_6t)} -45 e^{ik(x_{j-1}-c_6t)} \\
		                        & + 45e^{ik(x_{j+1}-c_6t)}-9e^{ik(x_{j+2}-c_6t)} + e^{ik(x_{j+3}-c_6t)} ]
	\end{aligned}
	\label{eq:wave_scheme}
\end{equation}
Now divided both side by $ e^{ik(x_j-c_6t)}$
\begin{equation}
	\begin{aligned}
		-ikc_6 & = -\frac{c}{60 \Delta x} [-e^{ik(x_{j-3}-x_j)} + 9e^{ik(x_{j-2}-x_j)} -45 e^{ik(x_{j-1}-x_j)}                                                                \\
		       & \quad + 45e^{ik(x_{j+1}-x_j)}-9e^{ik(x_{j+2}-x_j)} + e^{ik(x_{j+3}-x_j)} ]                                                                                   \\
		       & = -\frac{c}{60 \Delta x} \left [-e^{-3ik \Delta x} + 9e^{-2ik\Delta x} -45 e^{-ik \Delta x} + 45e^{ik\Delta x} -9e^{2ik \Delta x} + e^{3ik\Delta x} \right ]
	\end{aligned}
\end{equation}
We can now use the following formula $e^{ik \phi} - e^{-ik \phi} = 2i \sin (k \phi)$ to simplify the equation even further
\begin{equation}
	\begin{aligned}
		-ikc_6 & =  -\frac{c}{60 \Delta x} \left [ 2i \sin (3k \Delta x ) - 18i \sin (2k \Delta x) + 90i \sin (k \Delta x) \right] \\
		       & = -\frac{c}{30 \Delta x}i  \left [ \sin (3k \Delta x) - 9 \sin (2k \Delta x) + 45 \sin (k \Delta x) \right]       \\
	\end{aligned}
	\label{eq:pairs}
\end{equation}
Dividing by $-ik$ we end up with the wave speed for the 6th order approximation
\begin{equation}
	c_6(k) = c\frac{45 \sin (k \Delta x) - 9 \sin (2k \Delta x) + \sin (3k \Delta x) }{30 k\Delta x}
	\label{eq:wave_speed}
\end{equation}
\subsection*{Phase Error}
For the $2m$-order scheme we can define the phase error as
\begin{equation}
	e_m(k)=\left|\frac{u(x, t)-v(x, t)}{u(x, t)}\right|=\left|1-e^{i k\left(c-c_m\right) t}\right| \approx k t\left|c-c_m(k)\right|
	\label{eq:phase_error}
\end{equation}
Now using our result in \eqref{eq:wave_speed} we can define the phase error for a 6th order scheme as follows
\begin{equation}
	e_6(k, t)=kct\left|1- \frac{45 \sin (k \Delta x) - 9 \sin (2k \Delta x) + \sin (3k \Delta x) }{30 k\Delta x}\right|
	\label{eq:6phase_error}
\end{equation}
To measure the accuracy of the 6th scheme, let's introduce the number of grid points per waive length
\begin{equation}
	p = \frac{\lambda}{\Delta x}= \frac{2 \pi}{k \Delta x}
	\label{eq:n_gp}
\end{equation}
and the number of time the solution returns to itself, due to the periodicity
\begin{equation}
	\nu = \frac{ct}{\lambda}
	\label{eq:periodicity}
\end{equation}
Rewriting the phase error \eqref{eq:6phase_error} in terms of $p$ for $k$ and $\nu$ for $t$ results in
\begin{equation}
	\begin{aligned}
		e_6(p, \nu) & = \frac{2 \pi}{(\lambda/ \Delta x) \Delta x} c \frac{\nu \lambda}{c} \left|1- \frac{45 \sin (2 \pi p^{-1}) - 9 \sin (2 \cdot 2 \pi p^{-1}) + \sin (3 \cdot 2 \pi p^{-1}) }{(30 \cdot 2 \pi p^{-1})}\right| \\
		            & = 2 \pi \nu \left|1- \frac{45 \sin (2 \pi p^{-1}) - 9 \sin (4 \pi p^{-1}) + \sin (6 \pi p^{-1})}{(60 \pi p^{-1})}\right|                                                                                   \\
	\end{aligned}
	\label{eq:6phase_error_pnu}
\end{equation}
If we now perform a leading-order approximation ($p \rightarrow \infty$) this means the terms including $p^{-1}$ become small we can use the Taylor series expansion for $\sin$, which is given by the following formula
To perform a leading-order approximation as $p \to \infty$, we use the Taylor series expansion for sine:
\begin{equation}
	\sin(x) = \sum_{n=0}^{\infty} \frac{(-1)^n}{(2n + 1)!}x^{2n+1} = x - \frac{x^3}{3!} + \frac{x^5}{5!} - \frac{x^7}{7!} + \ldots \tag{30}
\end{equation}
e now expand each sine term in the numerator:
For $\sin(2\pi p^{-1})$:
\begin{equation}
	\sin(2\pi p^{-1}) = 2\pi p^{-1} - \frac{(2\pi p^{-1})^3}{3!} + \frac{(2\pi p^{-1})^5}{5!} - \frac{(2\pi p^{-1})^7}{7!} + \mathcal{O}(p^{-9})
\end{equation}
For $\sin(4\pi p^{-1})$:
\begin{equation}
	\sin(4\pi p^{-1}) = 4\pi p^{-1} - \frac{(4\pi p^{-1})^3}{3!} + \frac{(4\pi p^{-1})^5}{5!} - \frac{(4\pi p^{-1})^7}{7!} + \mathcal{O}(p^{-9})
\end{equation}
For $\sin(6\pi p^{-1})$:
\begin{equation}
	\sin(6\pi p^{-1}) = 6\pi p^{-1} - \frac{(6\pi p^{-1})^3}{3!} + \frac{(6\pi p^{-1})^5}{5!} - \frac{(6\pi p^{-1})^7}{7!} + \mathcal{O}(p^{-9})
\end{equation}
Now we calcualte the numerator for each term of each order.\newline
For the first order terms:
\begin{equation}
	45\left( 2 \pi p^{-1}\right) - 9\left(4 \pi p^{-1}\right) + \left(6 \pi p^{-1}\right) = (90 - 36 + 6) \pi p^{-1} = 60 \pi p^{-1}
\end{equation}
For the third order terms:
\begin{equation}
	45\left( 8 \pi^3 p^{-3}\right) - 9\left(64 \pi^3 p^{-3}\right) + \left(216 \pi^3 p^{-3}\right) = (360 - 576 + 216) \pi^3 p^{-3} = 0 \pi^3 p^{-3} = 0
\end{equation}
For the fifth order terms:
\begin{equation}
	45\left( 32 \pi^5 p^{-5}\right) - 9\left(1024 \pi^5 p^{-5}\right) + \left(7776 \pi^5 p^{-5}\right) = (1440 - 9216 + 7776) \pi^5 p^{-5} = 0 \pi^3 p^{-3} = 0
\end{equation}
and seventh order terms:
\begin{equation}
	\begin{aligned}
		 & 45\left(-\frac{128\pi^7}{5040}p^{-7}\right) - 9\left(-\frac{16384\pi^7}{5040}p^{-7}\right) + \left(-\frac{279936\pi^7}{5040}p^{-7}\right) \\
		 & = -\frac{45 \cdot 128\pi^7}{5040}p^{-7} + \frac{9 \cdot 16384\pi^7}{5040}p^{-7} - \frac{279936\pi^7}{5040}p^{-7}                          \\
		 & = -\frac{5760\pi^7}{5040}p^{-7} + \frac{147456\pi^7}{5040}p^{-7} - \frac{279936\pi^7}{5040}p^{-7}                                         \\
		 & = \frac{-5760 + 147456 - 279936}{5040}\pi^7p^{-7}                                                                                         \\
		 & = \frac{-138240}{5040}\pi^7p^{-7}                                                                                                         \\
		 & = \frac{-192}{7}\pi^7p^{-7}                                                                                                               \\
	\end{aligned}
\end{equation}
Now, substituting back into our original expression:
\begin{equation}
	\begin{aligned}
		e_6(p, \nu) & = 2 \pi \nu \left|1- \frac{60 \pi p^-{1}}{60 \pi p^-{1} } + \frac{\frac{-192}{7}\pi^7p^{-7}}{(60 \pi p^{-1})}\right| \\
		            & = 2\pi\nu \left|\frac{192}{420}\pi^6 p^{-6}\right|                                                                   \\
		            & = 2\pi\nu \left|\frac{16}{35}\pi^6 p^{-6}\right|                                                                     \\
	\end{aligned}
\end{equation}
We can remove the absolute brackets due to p being a large postive number, resulting in the final step:
\begin{equation}
	\begin{aligned}
		e_3(p, \nu) & = 2\pi\nu \frac{16}{35}\pi^6 p^{-6} = \frac{\pi \nu}{70} \left ( \frac{2\pi}{p}\right)^6 \\
	\end{aligned}
\end{equation}
We can use now the obtained equation to derive the number of points per wavelength required to ensure that the phase error is bounded by $\epsilon_3$.
\begin{equation}
	\begin{aligned}
		e_3(p, \nu)                               & = \frac{\pi \nu}{70} \left ( \frac{2\pi}{p} \right)^6 \\
		\frac{70 e_3(p, \nu)}{\pi \nu}            & =\left ( \frac{2\pi}{p} \right)^6                     \\
		\frac{70 e_3(p, \nu)}{\pi \nu}            & =\left ( \frac{2\pi}{p} \right)^6                     \\
		\sqrt[6]{\frac{70 e_3(p, \nu)}{\pi \nu} } & =  \frac{2\pi}{p}                                     \\
		\sqrt[6]{\frac{70 e_3(p, \nu)}{\pi \nu}}  & = \frac{2\pi}{p}                                      \\
		p                                         & = 2 \pi \sqrt[6]{\frac{\pi \nu}{70e_3(p, \nu)} }
	\end{aligned}
	\label{eq:bound}
\end{equation}
Hence the bound is given as
\begin{equation}
	p_3(\epsilon_p, \nu) \geq 2 \pi \sqrt[6]{\frac{\pi \nu}{70\epsilon_p} }
	\label{eq:bound_final}
\end{equation}



