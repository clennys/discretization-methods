\section{Fourier-Galerking Approximation: Exercise 1}
\subsection{Derivation}
We are considering the variable coefficient problem
\begin{equation}
	\frac{\partial u}{\partial t}+\sin (x) \frac{\partial u}{\partial x}=0
	\label{eq:pde}
\end{equation}
with periodic boundary condition.\\
We assume that $u(x,t)$ is periodic in $x \in [0, 2\pi]$ and use a truncated Fourier series to define
\begin{equation}
	u_N = \sum_{n=0}^N \hat{u}_n(t) \phi_n(x) = \sum_{n=-\frac{N}{2}}^{\frac{N}{2}} \hat{u}_n(t) e^{inx}
	\label{eq:un}
\end{equation}
where the continues expansion coefficients $\hat{u}_n$ are defined as (see Lecture 8)
\begin{equation}
	\hat{u}_n(t) = \frac{1}{2 \pi} \int_0^{2\pi} u(x, t) e^{-inx} dx .
	\label{eq:un_hat}
\end{equation}
We can now substitute $u_N$ into the PDE \eqref{eq:pde}:
\begin{equation}
	\frac{\partial u_N}{\partial t}+\sin (x) \frac{\partial u_N}{\partial x}=0
	\label{eq:un_pde}
\end{equation}
Computing each term
\begin{itemize}
	\item \textit{Time derivative}:
	      \begin{equation}
		      \frac{\partial u_N}{\partial t} = \sum_{n=-\frac{N}{2}}^{\frac{N}{2}} \frac{d \hat{u}_n}{dt} e^{inx}
		      \label{eq:time_der}
	      \end{equation}
	\item \textit{Spatial derivative}:
	      \begin{equation}
		      \frac{\partial u_N}{ \partial x} = \sin (x) \sum_{n=-\frac{N}{2}}^{\frac{N}{2}} \hat{u}_n \frac{d}{dx}\left [ e^{inx} \right] =  \sum_{n=-\frac{N}{2}}^{\frac{N}{2}} i n \hat{u}_n \sin(x)  e^{inx}
		      \label{eq:spat_der}
	      \end{equation}
\end{itemize}
The Residual is now given as follows
\begin{equation}
	R_N(x, t)=\frac{\partial u_N}{\partial t}-\mathcal{L} u_N = \frac{\partial u_N}{\partial t}+\sin (x) \frac{\partial u_N}{\partial x}
\end{equation}
and we need to enforce $\left(R_N, \psi_m\right)_w=0, \forall m \in[-N / 2, N / 2]$, where $\psi_m=\frac{1}{\gamma_m} \phi_m$, so that $\left(\phi_m, \psi_n\right)_w=\delta_{m n}$.
\begin{equation}
	\left(R_N, \psi_m\right)_w = \int_0^{2\pi} \left(\frac{\partial u_N}{\partial t}-\mathcal{L} u_N \right)\overline{\psi_m}wdx = \int_0^{2\pi} \left (  \frac{\partial u_N}{\partial t}+\sin (x) \frac{\partial u_N}{\partial x} \right) e^{-imx} = 0
	\label{eq:res}
\end{equation}
where $w$ is chosen to be $w(x) = 1$ so that
\begin{equation}
	(\psi_n, \psi_m)_w = \int_0^{2\pi} e^{inx} e^{-imx} = 2 \pi \delta_{nm} = \begin{cases}
		2 \pi & \text{if } n = m    \\
		0     & \text{if } n \neq m
	\end{cases}
	\label{eq:weight_f}
\end{equation}
so in this case $\psi_m = \frac{1}{2\pi}e^{-imx}$, thus $\psi_m(x) = \phi_m(x) = e^{imx}$. \\
If we now plugin the results of \eqref{eq:time_der} and \eqref{eq:spat_der} into \eqref{eq:res} we get
\begin{equation}
	\begin{aligned}
		\left(R_N, \psi_m\right)_w & = \int_0^{2\pi} \left (  \sum_{n=-\frac{N}{2}}^{\frac{N}{2}} \frac{d \hat{u}_n}{dt} e^{inx} +   \sum_{n=-\frac{N}{2}}^{\frac{N}{2}} i n \hat{u}_n \sin(x)  e^{inx} \right) e^{-imx} \\
	\end{aligned}
	\label{eq:res_der}
\end{equation}
First lets simplify the first term inside the integral
\begin{equation}
	\int_0^{2 \pi} \sum_{n=-\frac{N}{2}}^{\frac{N}{2}}  \frac{d \hat{u}_n}{dt} e^{inx}  e^{-imx} dx = \sum_{n=-\frac{N}{2}}^{\frac{N}{2}}  \frac{d \hat{u}_n}{dt} \int_0^{2 \pi} e^{i(n-m)x} = \sum_{n=-\frac{N}{2}}^{\frac{N}{2}}  \frac{d \hat{u}_n}{dt} 2 \pi \delta_{nm} = 2 \pi \frac{d \hat{u}_m}{dt}
	\label{eq:res_first}
\end{equation}
For the second term we can rewrite sine as
\begin{equation}
	\sin(x) = \frac{e^{ix} - e^{-ix}}{2i}
	\label{eq:sin_exp}
\end{equation}
and then substitute it into the second term
\begin{equation}
	\begin{aligned}
		\int_0^{2\pi} \sum_{n=-\frac{N}{2}}^{\frac{N}{2}} i n \hat{u}_n \sin(x)  e^{inx} e^{-imx} dx & =  \sum_{n=-\frac{N}{2}}^{\frac{N}{2}} i n \hat{u}_n  \int_0^{2\pi} \frac{e^{ix} - e^{-ix}}{2i} e^{i(n-m)x} dx       \\
		                                                                                             & =   \sum_{n=-\frac{N}{2}}^{\frac{N}{2}} i n \hat{u}_n  \int_0^{2\pi} \frac{e^{i(n-m + 1)x} - e^{i(n-m -1) x}}{2i} dx \\
		\label{eq:res_second}
	\end{aligned}
\end{equation}
using the orthogonality property again, this integral becomes non-zero only when:
\begin{equation}
	\begin{aligned}
		n - m + 1 & = 0 \Rightarrow n = m -1   \\
		n - m - 1 & = 0 \Rightarrow n = m  + 1
	\end{aligned}
	\label{eq:res_second_non_zero}
\end{equation}
Therefore the second term becomes
\begin{equation}
	i(m-1)\hat{u}_{m-1} \frac{\int_0^{2\pi} e^0 dx}{2i} - i(m+1) \hat{u}_{m+1} \frac{\int_0^{2\pi} e^0 dx}{2i} = \pi \left [ (m - 1)\hat{u}_{m-1} - (m+1) \hat{u}_{m+1}  \right]
	\label{eq:res_second_final}
\end{equation}
Now we can combine both terms, resulting in
\begin{equation}
	2 \pi \frac{d \hat{u}_m}{dt} + \pi \left [ (m - 1)\hat{u}_{m-1} - (m+1) \hat{u}_{m+1}  \right] = 0
	\label{eq:res_eqs}
\end{equation}
Rearranging this result leads to our systems of ODEs for the Fourier-Galerking approximation
\begin{equation}
	\frac{d \hat{u}_m}{dt} = \frac{ (m + 1)\hat{u}_{m+1} - (m-1) \hat{u}_{m-1}}{2}
	\label{eq:FG_ODEs}
\end{equation}
and the initial condition is given as 
\begin{equation}
	\hat{u}_m (0) = \frac{1}{2\pi} \int_0^{2\pi}g(x)e^{-imx}dx
	\label{eq:init}
\end{equation}
\subsection{Is $P_N u = u_N?}
If we apply the Fourier-Galerking Approximation to the equation satisfied by the exact solution $u$, we get
\begin{equation}
	\frac{d \hat{u}_m}{dt} = \frac{ (m + 1)\hat{u}_{m+1} - (m-1) \hat{u}_{m-1}}{2} + \epsilon_T
	\label{eq:trunc_err}
\end{equation}
where $\epsilon_T$ is the truncation error. In order for the $P_N u = u_N$ to hold $\epsilon_T$ has to be $0$.
\begin{equation}
	P_N\left(\frac{\partial u}{\partial t} + \sin(x)\frac{\partial u}{\partial x}\right) = \frac{\partial P_N u}{\partial t} + P_N\left(\sin(x)\frac{\partial u}{\partial x}\right) = P_N(0) = 0
	\label{eq:}
\end{equation}
The Problem is that in general
\begin{equation}
	P_N\left(\sin(x)\frac{\partial u}{\partial x}\right) \neq \sin(x) \frac{\partial P_N u}{\partial x}
	\label{eq:prob}
\end{equation}
because when we are multiplying $\sin(x) = \frac{e^{ix} - e^{-ix}}{2i}$ with $\frac{\partial u}{dx} = \sum_{-\infty}^{\infty} i n \hat{u}_n e^{inx}$ it creates frequencies that are outside our truncated space, created by the projection into finite space by $P_N$.

